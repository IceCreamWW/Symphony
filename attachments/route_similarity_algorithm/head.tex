\documentclass[a4paper]{article}
\usepackage{xeCJK}
\usepackage{amsmath}
\usepackage{amsthm}
\usepackage{setspace}
\usepackage{enumerate}
\usepackage{mathrsfs}
\usepackage{amsfonts}
\usepackage{graphicx}
\usepackage{caption}
\usepackage{geometry}
\usepackage{indentfirst}
\usepackage{amssymb}
\usepackage{algorithm}
\usepackage{algorithmicx}
\usepackage{algpseudocode}

\setlength{\parindent}{2em}%设置缩进
\renewcommand{\baselinestretch}{1.5}%1.5倍行间距
\setlength{\parskip}{0.5\baselineskip}%1.5倍段间距
\geometry{left=2.5cm,right=2.5cm,top=2.5cm,bottom=2.5cm}%页边距
\captionsetup[figure]{labelfont={bf},name={图},labelsep=period}
%\setmainfont{Times New Roman}
\setCJKmainfont[ItalicFont={楷体},BoldFont={华文中宋}]{宋体}

%字号设置
\newcommand{\chuhao}{\fontsize{42pt}{\baselineskip}\selectfont}
\newcommand{\xiaochuhao}{\fontsize{36pt}{\baselineskip}\selectfont}
\newcommand{\yihao}{\fontsize{28pt}{\baselineskip}\selectfont}
\newcommand{\xiaoyihao}{\fontsize{24pt}{\baselineskip}\selectfont}
\newcommand{\erhao}{\fontsize{21pt}{\baselineskip}\selectfont}
\newcommand{\xiaoerhao}{\fontsize{18pt}{\baselineskip}\selectfont}
\newcommand{\sanhao}{\fontsize{15.75pt}{\baselineskip}\selectfont}
\newcommand{\sihao}{\fontsize{14pt}{\baselineskip}\selectfont}
\newcommand{\xiaosihao}{\fontsize{12pt}{\baselineskip}\selectfont}
\newcommand{\wuhao}{\fontsize{10.5pt}{\baselineskip}\selectfont}
\newcommand{\xiaowuhao}{\fontsize{9pt}{\baselineskip}\selectfont}
\newcommand{\liuhao}{\fontsize{7.875pt}{\baselineskip}\selectfont}
\newcommand{\qihao}{\fontsize{5.25pt}{\baselineskip}\selectfont}

% 重定义字体命令
\newcommand{\song}{\CJKfamily{song}}    % 宋体   (Windows自带simsun.ttf)
\newcommand{\fs}{\CJKfamily{fs}}        % 仿宋体 (华天字库htfs.ttf)
\newcommand{\kai}{\CJKfamily{kai}}      % 楷体   (华天字库htkai.ttf)
\newcommand{\hei}{\CJKfamily{hei}}      % 黑体   (Windows自带simhei.ttf)
\newcommand{\li}{\CJKfamily{li}}        % 隶书   (Windows自带simli.ttf)
\newcommand{\you}{\CJKfamily{you}}      % 幼圆体 (Windows自带simyou.ttf)

\allowdisplaybreaks[4]
\newtheorem{definition}{定义}
\newtheorem{thm}{定理}
\newtheorem{prop}{命题}