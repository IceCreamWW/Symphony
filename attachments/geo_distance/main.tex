\documentclass[a4paper]{article}
\usepackage{xeCJK}
\usepackage{amsmath}
\usepackage{amsthm}
\usepackage{setspace}
\usepackage{enumerate}
\usepackage{mathrsfs}
\usepackage{amsfonts}
\usepackage{graphicx}
\usepackage{caption}
\usepackage{geometry}
\usepackage{indentfirst}
\usepackage{amssymb}
\usepackage{algorithm}
\usepackage{algorithmicx}
\usepackage{algpseudocode}

\setlength{\parindent}{2em}%设置缩进
\renewcommand{\baselinestretch}{1.5}%1.5倍行间距
\setlength{\parskip}{0.5\baselineskip}%1.5倍段间距
\geometry{left=2.5cm,right=2.5cm,top=2.5cm,bottom=2.5cm}%页边距
\captionsetup[figure]{labelfont={bf},name={图},labelsep=period}
%\setmainfont{Times New Roman}
\setCJKmainfont[ItalicFont={楷体},BoldFont={华文中宋}]{宋体}

%字号设置
\newcommand{\chuhao}{\fontsize{42pt}{\baselineskip}\selectfont}
\newcommand{\xiaochuhao}{\fontsize{36pt}{\baselineskip}\selectfont}
\newcommand{\yihao}{\fontsize{28pt}{\baselineskip}\selectfont}
\newcommand{\xiaoyihao}{\fontsize{24pt}{\baselineskip}\selectfont}
\newcommand{\erhao}{\fontsize{21pt}{\baselineskip}\selectfont}
\newcommand{\xiaoerhao}{\fontsize{18pt}{\baselineskip}\selectfont}
\newcommand{\sanhao}{\fontsize{15.75pt}{\baselineskip}\selectfont}
\newcommand{\sihao}{\fontsize{14pt}{\baselineskip}\selectfont}
\newcommand{\xiaosihao}{\fontsize{12pt}{\baselineskip}\selectfont}
\newcommand{\wuhao}{\fontsize{10.5pt}{\baselineskip}\selectfont}
\newcommand{\xiaowuhao}{\fontsize{9pt}{\baselineskip}\selectfont}
\newcommand{\liuhao}{\fontsize{7.875pt}{\baselineskip}\selectfont}
\newcommand{\qihao}{\fontsize{5.25pt}{\baselineskip}\selectfont}

% 重定义字体命令
\newcommand{\song}{\CJKfamily{song}}    % 宋体   (Windows自带simsun.ttf)
\newcommand{\fs}{\CJKfamily{fs}}        % 仿宋体 (华天字库htfs.ttf)
\newcommand{\kai}{\CJKfamily{kai}}      % 楷体   (华天字库htkai.ttf)
\newcommand{\hei}{\CJKfamily{hei}}      % 黑体   (Windows自带simhei.ttf)
\newcommand{\li}{\CJKfamily{li}}        % 隶书   (Windows自带simli.ttf)
\newcommand{\you}{\CJKfamily{you}}      % 幼圆体 (Windows自带simyou.ttf)
\renewcommand\refname{参考文献}

\allowdisplaybreaks[4]
\newtheorem{definition}{定义}
\newtheorem{thm}{定理}
\newtheorem{prop}{命题}
\title{\textbf{在MySQL数据库中由经纬度计算最近的点}}
\author{}
\date{}
\begin{document}
\maketitle
\section{问题背景}
给定点集$S=\{(x,y)\mid x\in[-90, 90], y\in[-180,
180]\}$,其中每一个点$(x, y)\in
S$都表示三维球面上的一个点,并且$x$表示纬度,$y$表示经度。假设$S$存储在MySQL数据库sites中,并
且sites的列为latitude,longitude,以及route\_id,其中$\text{route\_id}\in\{0,1,2,\dots\}$,且$S$中每
个元素都有不同的route\_id。现有点$a=(x_0,y_0)\;(x_0\in[-90,90],y_0\in[-180,180])$,往求解在球面距离
下$S$中距离$a$最短的点$b\in R$。

\section{距离的计算}
对于三维球面坐标系下已知经度纬度时在球面上的距离的计算,我们有如下定理。

\begin{thm}[三维球体表面的距离]
  设两个点$s_1,s_2$在半径为$R$的三维球体表面,并且$s_1$的纬度为$\phi_1$,$s_2$的纬度
  为$\phi_2$,$s_1,s_2$的纬度差为$\Delta\phi$,经度差为$\Delta\lambda$,则$s_1,s_2$间的距离$d$为:
\begin{displaymath}
  d=\mathrm{haversin}(\frac{d}{R})=\mathrm{haversin}(\Delta\phi)+\cos(\phi_1)\cos(\phi_2)\mathrm{haversin}(\Delta\lambda)
\end{displaymath}
其中,$\mathrm{haversion}(\theta)=\frac{\mathrm{versin}(\theta)}{2}=\sin^2(\frac{\theta}{2})$,
$\mathrm{versin}(\theta)=1-\cos(\theta)=2\sin^2(\frac{\theta}{2})$。
\end{thm}

在MySQL数据库中,若第一个点的纬度为\texttt{orig.latitude},经度为\texttt{orig.longitude},第二个点的
纬度为\texttt{dest.latitude},经度为\texttt{dest.longitude},则可用如下命令计算两个点之间在球表面的距
离:

\begin{quote}
  \texttt{3956 * 2 * ASIN(SQRT(\\
    POWER(SIN((orig.latitude - dest.latitude) * pi() / 180 / 2), 2)\\
    + COS(orig.latitude * pi() / 180) * COS(dest.latitude * pi() / 180)\\
    * POWER(SIN((orig.longitude - dest.longitude) * pi() / 180 / 2), 2)))\\
    as distance}
\end{quote}

\section{计算最近点的MySQL语句}
首先,根据上一节中距离的计算公式,我们可以直接得到一个直接的命令来求解最近的点:

\begin{quote}
  \texttt{SELECT *, 3956 * 2 * ASIN(SQRT(\\
    POWER(SIN((@orig\_latitude - dest.latitude) * pi() / 180 / 2), 2)\\
    + COS(@orig\_latitude * pi() / 180) * COS(dest.latitude * pi() / 180)\\
    * POWER(SIN((@orig\_longitude - dest.longitude) * pi() / 180 / 2), 2)))\\
    as distance\\
    FROM sites dest\\
    ORDER BY distance limit 1;}
\end{quote}
其中,\texttt{@orig\_latitude},\texttt{@orig\_longitude}的值分别为点$a$的纬度与经度值。

以上命令在数据量较小时有较好的表现,但是当数据量增多时,则查询速度明显较慢。因此,在数据量较大时,我
们需要采取一些方法提高计算的性能。

一个可行的改进即为限定搜索的区域,即划定一个边长为$2d\;\mathrm{km}$的区域,仅在此区域内计算距离进行搜
索。首先,需要计算这个正方形区域主对角线两个顶点的经纬度。我们知道,$1\;\mathrm{latitude}$对应的距离
约为$111\;\mathrm{km}$,$1\;\mathrm{longitude}$对应的距离约
为$111\times\cos(\mathrm{latitude})\;\mathrm{km}$。因此,我们有计算正方形两个顶点的经纬度的MySQL命令:
\begin{quote}
  \texttt{SET @lon1 = @orig\_longitude - @d / ABS(COS(RADIANS(@orig\_latitude)) * 111);\\
  SET @lon2 = @orig\_longitude + @d / ABS(COS(RADIANS(@orig\_latitide)) * 111);\\
  SET @lat1 = @orig\_latitude - (@d / 111);\\
  SET @lat2 = @orig\_latitide + (@d / 111);}
\end{quote}

由此,可将查询最近点的MySQL命令改写为:

\begin{quote}
  \texttt{SELECT *, 3956 * 2 * ASIN(SQRT(\\
    POWER(SIN((@orig\_latitude - dest.latitude) * pi() / 180 / 2), 2)\\
    + COS(@orig\_latitude * pi() / 180) * COS(dest.latitude * pi() / 180)\\
    * POWER(SIN((@orig\_longitude - dest.longitude) * pi() / 180 / 2), 2)))\\
    as distance\\
    FROM sites dest\\
    WHERE dest.longitude BETWEEN @lon1 AND @lon2\\
    AND dest.lantitude BETWEEN @lan1 AND @lan2\\
    ORDER BY distance limit 1;}
\end{quote}

由实验可知,该查询语句能够大大提高查询速度。最后,还可将上述过程利用MySQL中的存储过程(Storage
Procedure)加以实现,以进一步提高查询的速度。该存储过程的定义命令如下:

\begin{quote}
  \texttt{CREATE PROCEDURE snn(IN lat, IN lon)\\
    BEGIN\\
    DECLARE lon1 FLOAT; DECLARE lon2 FLOAT;\\
    DECLARE lat1 FLOAT; DECLARE lat2 FLOAT;\\
    --calculate longitude and latitude of rectangle\\
    set dist = 100;\\
    set lon1 = lon - dist / ABS(COS(REDIANS(lat)) * 111);\\
    set lon2 = lon + dist / ABS(COS(REDIANS(lat)) * 111);\\
    set lat1 = lat - (dist / 111);\\
    set lat2 = lat + (disy / 111);\\
    --run query\\
    SELECT *, 3956 * 2 * ASIN(SQRT(\\
    POWER(SIN((@orig\_latitude - dest.latitude) * pi() / 180 / 2), 2)\\
    + COS(@orig\_latitude * pi() / 180) * COS(dest.latitude * pi() / 180)\\
    * POWER(SIN((@orig\_longitude - dest.longitude) * pi() / 180 / 2), 2)))\\
    as distance\\
    FROM sites dest\\
    WHERE dest.longitude BETWEEN lon1 AND lon2\\
    AND dest.lantitude BETWEEN lan1 AND lan2\\
    ORDER BY distance limit 1;\\
    END} 
\end{quote}

由此,我们只需通过命令\texttt{CALL snn(@orig\_latitude, @orig\_longitude)}即可较为高效的查询距离$a$最
近的点$b\in S$。
\end{document}

%%% Local Variables:
%%% mode: latex
%%% TeX-master: t
%%% End: